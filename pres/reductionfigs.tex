
% Figures that illustrate optimisation effort of Reduction kernels 

%% \newcommand{\arrayLength}[1]{%
%%   \setcounter{arraycard}{0}%
%%   \foreach \x in #1{%
%%     \stepcounter{arraycard}%
%%   }%
%%   \the\value{arraycard}%
%% }  

% smrow{x}{y}{data}{largest_index}{identifier}
\newcommand{\smRow}[5] { 
  
 %                                       (INSANE!) 
  \node [draw, fill=gray!30,anchor=west] (#50) at (#1,#2) {\pgfmathparse{#3[0]}\pgfmathresult};   
  
  \ifnum#4>0
    \foreach[count=\i] \j in {1,...,#4} {
      \pgfmathtruncatemacro{\n}{int(\i) - 1};
      \pgfmathtruncatemacro{\m}{int(\i)};
      \node [draw, fill=gray!30,right=0cm of #5\n,anchor=west] (#5\m) {\pgfmathparse{#3[\j]}\pgfmathresult};   % {\j};   
    } 
  \fi
}


% smRowTiny{x}{y}{data}{largest_index}{identifier} 
\newcommand{\smRowTiny}[5] { 
  
 %                                       
  \node [draw, fill=gray!30,anchor=west,inner sep=1pt] (#50) at (#1,#2) {\tiny\pgfmathparse{#3[0]}\pgfmathresult};   
  
  \ifnum#4>0
    \foreach[count=\i] \j in {1,...,#4} {
      \pgfmathtruncatemacro{\n}{int(\i) - 1};
      \pgfmathtruncatemacro{\m}{int(\i)};
      \node [draw, fill=gray!30,right=0cm of #5\n,anchor=west,inner sep=1pt] (#5\m) {\tiny\pgfmathparse{#3[\j]}\pgfmathresult};   % {\j};   
    } 
  \fi
}

\newcommand{\seqRed}[3] { 
 \tikzset{sstyle/.style={anchor=west,draw,shape=circle,fill,inner sep=1pt}};
 \tikzset{cstyle/.style={anchor=west,draw,shape=circle,inner sep=0pt}};
 %\tikzstyle{every node}=[anchor=west,draw,shape=circle,fill,inner sep=1pt]

 \node [sstyle](r1) at (0+ #1,#2) {};
 \node [sstyle,right=.5mm of r1] (r2) {};
 \node [sstyle,right=.5mm of r2] (r3) {};
 \node [sstyle,right=.5mm of r3] (r4) {};

 
 \path [draw] (r1) -- (r2) ;
 \path [draw] (r2) -- (r3) ;
 \path [draw] (r3) -- (r4) ;
 
 % connectors 

 %\tikzstyle{every node}=[anchor=west,draw,shape=circle,inner sep=0pt]

 \node [cstyle,above=.5mm of r1] (#3c1) {};
 \node [cstyle,above=.5mm of r2] (#3c2) {};
 \node [cstyle,above=.5mm of r3] (#3c3) {};
 \node [cstyle,above=.5mm of r4] (#3c4) {};

 \node [cstyle,below=.5mm of r4] (#3c5) {};

 % wires 

 \path [draw] (r1) -- (#3c1) ;
 \path [draw] (r2) -- (#3c2) ;
 \path [draw] (r3) -- (#3c3) ;
 \path [draw] (r4) -- (#3c4) ;
 \path [draw] (r4) -- (#3c5) ;


}



%\foreach[count=\i] \j in {1,1,1,1,1,1,1,1} { 
%  \pgfmathtruncatemacro{\n}{int(\i)};
%  \node[draw=black] (a\n) at (\i,-2) {\j};
%}



\begin{figure} 

% --------------------------------------------------------------------------- 
% Pair fmap 
% ---------------------------------------------------------------------------
\begin{minipage}{.45\linewidth} 
\begin{tikzpicture}[scale = 0.40]

\smRow{0}{0}{{0,1,2,3,4,5,6,7}}{7}{a}

\smRow{0}{-2}{{1,5,9,13}}{3}{b}

\smRow{0}{-4}{{6,22}}{1}{c}

\smRow{0}{-6}{{28,0}}{0}{d}  % HACK

%% \smRow{0}{0}{{0,1,2,3,4,5,6,7}}{7}{a}

%% \smRow{1.75}{-2}{{1,5,9,13}}{3}{b}

%% \smRow{2.5}{-4}{{6,22}}{1}{c}

%% \smRow{3}{-6}{{28,0}}{0}{d}  % HACK

% connections 


\path[->,draw=black] (a0.south) -- (b0.north) ; 
\path[->,draw=black] (a1.south) -- (b0.north) ; 

\path[->,draw=black] (a2.south) -- (b1.north) ; 
\path[->,draw=black] (a3.south) -- (b1.north) ; 

\path[->,draw=black] (a4.south) -- (b2.north) ; 
\path[->,draw=black] (a5.south) -- (b2.north) ; 

\path[->,draw=black] (a6.south) -- (b3.north) ; 
\path[->,draw=black] (a7.south) -- (b3.north) ; 


\path[->,draw=black] (b0.south) -- (c0.north) ; 
\path[->,draw=black] (b1.south) -- (c0.north) ; 

\path[->,draw=black] (b2.south) -- (c1.north) ; 
\path[->,draw=black] (b3.south) -- (c1.north) ; 

\path[->,draw=black] (c0.south) -- (d0.north) ; 
\path[->,draw=black] (c1.south) -- (d0.north) ; 


\end{tikzpicture} 

\end{minipage}
\hfill%hspace{2mm}%
% --------------------------------------------------------------------------- 
% Halve zip  
% ---------------------------------------------------------------------------
\begin{minipage}{.45\linewidth} 
\begin{tikzpicture}[scale = 0.40]

\smRow{0}{0}{{0,1,2,3,4,5,6,7}}{7}{a}

\smRow{0}{-2}{{4,6,8,10}}{3}{b}

\smRow{0}{-4}{{12,16}}{1}{c}

\smRow{0}{-6}{{28,0}}{0}{d}  % HACK

%% \smRow{0}{0}{{0,1,2,3,4,5,6,7}}{7}{a}

%% \smRow{1.75}{-2}{{4,6,8,10}}{3}{b}

%% \smRow{2.5}{-4}{{12,16}}{1}{c}

%% \smRow{3.25}{-6}{{28,0}}{0}{d}  % HACK

% connections 


\path[->,draw=black] (a0.south) -- (b0.north) ; 
\path[->,draw=black] (a4.south) -- (b0.north) ; 

\path[->,draw=black] (a1.south) -- (b1.north) ; 
\path[->,draw=black] (a5.south) -- (b1.north) ; 

\path[->,draw=black] (a2.south) -- (b2.north) ; 
\path[->,draw=black] (a6.south) -- (b2.north) ; 

\path[->,draw=black] (a3.south) -- (b3.north) ; 
\path[->,draw=black] (a7.south) -- (b3.north) ; 


\path[->,draw=black] (b0.south) -- (c0.north) ; 
\path[->,draw=black] (b2.south) -- (c0.north) ; 

\path[->,draw=black] (b1.south) -- (c1.north) ; 
\path[->,draw=black] (b3.south) -- (c1.north) ; 

\path[->,draw=black] (c0.south) -- (d0.north) ; 
\path[->,draw=black] (c1.south) -- (d0.north) ; 



\end{tikzpicture} 

\end{minipage}

%% \caption{\emph{Left:} {\tt evenOdds} - {\tt zipWith} reduction, leads to uncoalesced memory accesses.\newline
%% \emph{Right:} {\tt halve} - {\tt zipWith} reduction, leads to coalesced memory accesses.
%% This coalescing is most important during the very first phase, when data is 
%% read from global memory.} 

\label{fig:red1}
\end{figure} 



% ---------------------------------------------------------------------------
% Sequential reductions 
% ---------------------------------------------------------------------------


%% \begin{figure} 

%% \begin{tikzpicture}[scale = 0.40]

%% \smRowTiny{0}{{0,1,2,3,4,5,6,7,8,9,10,11,12,13,14,15,16,17,18,19,20,21,22,23,24,25,26,27,28,29,30,31}}{31}{a}

%% \seqRed{0}{-2}{redA}
%% \seqRed{2}{-2}{redB}
%% \seqRed{4}{-2}{redC}
%% \seqRed{6}{-2}{redD}
%% \seqRed{8}{-2}{redE}
%% \seqRed{10}{-2}{redF}
%% \seqRed{12}{-2}{redG}
%% \seqRed{14}{-2}{redH}

%% % connections 
%% \foreach[count=\i] \j in {redAc1,redAc2,redAc3,redAc4, 
%%                           redBc1,redBc2,redBc3,redBc4, 
%%                           redCc1,redCc2,redCc3,redCc4, 
%%                           redDc1,redDc2,redDc3,redDc4, 
%%                           redEc1,redEc2,redEc3,redEc4, 
%%                           redFc1,redFc2,redFc3,redFc4, 
%%                           redGc1,redGc2,redGc3,redGc4, 
%%                           redHc1,redHc2,redHc3,redHc4} {
%%   \pgfmathtruncatemacro{\n}{int(\i) - 1};
%%   \path[draw=black] (a\n.south) -- (\j.north) ; 
%% %\path[draw=black] (a1.south) -- (redAc2.north) ; 
%% %\path[draw=black] (a2.south) -- (redAc3.north) ; 
%% %\path[draw=black] (a3.south) -- (redAc4.north) ; 
%% } 

%% %next row of sM

%% \smRowTiny{-4}{{6,22,38,54,70,86,102,118}}{7}{b}

%% %connections 

%% \path[draw=black] (redAc5) -- (b0.north); 
%% \path[draw=black] (redBc5) -- (b1.north); 
%% \path[draw=black] (redCc5) -- (b2.north); 
%% \path[draw=black] (redDc5) -- (b3.north); 
%% \path[draw=black] (redEc5) -- (b4.north); 
%% \path[draw=black] (redFc5) -- (b5.north); 
%% \path[draw=black] (redGc5) -- (b6.north); 
%% \path[draw=black] (redHc5) -- (b7.north); 



%% %% \path[->,draw=black] (a0.south) -- (b0.north) ; 
%% %% \path[->,draw=black] (a1.south) -- (b0.north) ; 

%% %% \path[->,draw=black] (a2.south) -- (b1.north) ; 
%% %% \path[->,draw=black] (a3.south) -- (b1.north) ; 

%% %% \path[->,draw=black] (a4.south) -- (b2.north) ; 
%% %% \path[->,draw=black] (a5.south) -- (b2.north) ; 

%% %% \path[->,draw=black] (a6.south) -- (b3.north) ; 
%% %% \path[->,draw=black] (a7.south) -- (b3.north) ; 


%% %% \path[->,draw=black] (b0.south) -- (c0.north) ; 
%% %% \path[->,draw=black] (b1.south) -- (c0.north) ; 

%% %% \path[->,draw=black] (b2.south) -- (c1.north) ; 
%% %% \path[->,draw=black] (b3.south) -- (c1.north) ; 

%% %% \path[->,draw=black] (c0.south) -- (d0.north) ; 
%% %% \path[->,draw=black] (c1.south) -- (d0.north) ; 


%% \end{tikzpicture} 


%% \caption{Adding sequential reductions like this, reintroduces memory coalescing issues. 
%%  Consecutive threads nolonger access consecutive memory locations.} 

%% \label{fig:red1}
%% \end{figure} 




%% %%%%%%%%%%%%%%%%%%%%%%%%%%%%%%%%%%%%%%%%%%%%%%%%%%%%%%%%%%%%%%%%%%%%%%%%%%%
%%
%%  REINTRODUCE COALESCING
%%
%% %%%%%%%%%%%%%%%%%%%%%%%%%%%%%%%%%%%%%%%%%%%%%%%%%%%%%%%%%%%%%%%%%%%%%%%%%%%



\begin{figure} 

\begin{minipage}{0.45\linewidth}
\begin{tikzpicture}[scale = 0.40]

%\smRowTiny{0}{{0,1,2,3,4,5,6,7,8,9,10,11,12,13,14,15,16,17,18,19,20,21,22,23,24,25,26,27,28,29,30,31}}{31}{a}
\smRowTiny{0}{0}{{0,1,2,3,4,5,6,7,8,9,10,11,12,13,14,15}}{15}{a}

\seqRed{0}{-2}{redA}
\seqRed{2}{-2}{redB}
\seqRed{4}{-2}{redC}
\seqRed{6}{-2}{redD}
%\seqRed{8}{-2}{redE}
%\seqRed{10}{-2}{redF}
%\seqRed{12}{-2}{redG}
%\seqRed{14}{-2}{redH}

% connections 
\foreach[count=\i] \j in {redAc1,redAc2,redAc3,redAc4, 
                          redBc1,redBc2,redBc3,redBc4, 
                          redCc1,redCc2,redCc3,redCc4, 
                          redDc1,redDc2,redDc3,redDc4} {
                         
  \pgfmathtruncatemacro{\n}{int(\i) - 1};
  \path[draw=black] (a\n.south) -- (\j.north) ; 
} 

%next row of sM

%\smRowTiny{-4}{{6,22,38,54,70,86,102,118}}{7}{b}
\smRowTiny{0}{-4}{{6,22,38,54}}{3}{b}
%\smRowTiny{3}{-4}{{6,22,38,54}}{3}{b}


%connections 

\path[draw=black] (redAc5) -- (b0.north); 
\path[draw=black] (redBc5) -- (b1.north); 
\path[draw=black] (redCc5) -- (b2.north); 
\path[draw=black] (redDc5) -- (b3.north); 
%\path[draw=black] (redEc5) -- (b4.north); 
%\path[draw=black] (redFc5) -- (b5.north); 
%\path[draw=black] (redGc5) -- (b6.north); 
%\path[draw=black] (redHc5) -- (b7.north); 



%% \path[->,draw=black] (a0.south) -- (b0.north) ; 
%% \path[->,draw=black] (a1.south) -- (b0.north) ; 

%% \path[->,draw=black] (a2.south) -- (b1.north) ; 
%% \path[->,draw=black] (a3.south) -- (b1.north) ; 

%% \path[->,draw=black] (a4.south) -- (b2.north) ; 
%% \path[->,draw=black] (a5.south) -- (b2.north) ; 

%% \path[->,draw=black] (a6.south) -- (b3.north) ; 
%% \path[->,draw=black] (a7.south) -- (b3.north) ; 


%% \path[->,draw=black] (b0.south) -- (c0.north) ; 
%% \path[->,draw=black] (b1.south) -- (c0.north) ; 

%% \path[->,draw=black] (b2.south) -- (c1.north) ; 
%% \path[->,draw=black] (b3.south) -- (c1.north) ; 

%% \path[->,draw=black] (c0.south) -- (d0.north) ; 
%% \path[->,draw=black] (c1.south) -- (d0.north) ; 


\end{tikzpicture} 

%\label{fig:red1}

%\end{figure} 
\end{minipage}\hfill%\hspace{2mm}%
\begin{minipage}{0.45\linewidth}
%\begin{figure} 
%
\begin{tikzpicture}[scale = 0.40]

%\smRowTiny{0}{{0,1,2,3,4,5,6,7,8,9,10,11,12,13,14,15,16,17,18,19,20,21,22,23,24,25,26,27,28,29,30,31}}{31}{a}
\smRowTiny{0}{0}{{0,1,2,3,4,5,6,7,8,9,10,11,12,13,14,15}}{15}{a}

\seqRed{0}{-2}{redA}
\seqRed{2}{-2}{redB}
\seqRed{4}{-2}{redC}
\seqRed{6}{-2}{redD}
%\seqRed{8}{-2}{redE}
%\seqRed{10}{-2}{redF}
%\seqRed{12}{-2}{redG}
%\seqRed{14}{-2}{redH}

% connections 
\foreach[count=\i] \j in {redAc1,redBc1,redCc1,redDc1, 
                          redAc2,redBc2,redCc2,redDc2, 
                          redAc3,redBc3,redCc3,redDc3, 
                          redAc4,redBc4,redCc4,redDc4} {
                         
  \pgfmathtruncatemacro{\n}{int(\i) - 1};
  \path[draw=black] (a\n.south) -- (\j.north) ; 
} 

%next row of sM

%\smRowTiny{-4}{{6,22,38,54,70,86,102,118}}{7}{b}
%\smRowTiny{0}{-4}{{6,22,38,54}}{3}{b}
\smRowTiny{0}{-4}{{24,28,32,46}}{3}{b}
%\smRowTiny{3}{-4}{{6,22,38,54}}{3}{b}

%connections 

\path[draw=black] (redAc5) -- (b0.north); 
\path[draw=black] (redBc5) -- (b1.north); 
\path[draw=black] (redCc5) -- (b2.north); 
\path[draw=black] (redDc5) -- (b3.north); 
%\path[draw=black] (redEc5) -- (b4.north); 
%\path[draw=black] (redFc5) -- (b5.north); 
%\path[draw=black] (redGc5) -- (b6.north); 
%\path[draw=black] (redHc5) -- (b7.north); 



%% \path[->,draw=black] (a0.south) -- (b0.north) ; 
%% \path[->,draw=black] (a1.south) -- (b0.north) ; 

%% \path[->,draw=black] (a2.south) -- (b1.north) ; 
%% \path[->,draw=black] (a3.south) -- (b1.north) ; 

%% \path[->,draw=black] (a4.south) -- (b2.north) ; 
%% \path[->,draw=black] (a5.south) -- (b2.north) ; 

%% \path[->,draw=black] (a6.south) -- (b3.north) ; 
%% \path[->,draw=black] (a7.south) -- (b3.north) ; 


%% \path[->,draw=black] (b0.south) -- (c0.north) ; 
%% \path[->,draw=black] (b1.south) -- (c0.north) ; 

%% \path[->,draw=black] (b2.south) -- (c1.north) ; 
%% \path[->,draw=black] (b3.south) -- (c1.north) ; 

%% \path[->,draw=black] (c0.south) -- (d0.north) ; 
%% \path[->,draw=black] (c1.south) -- (d0.north) ; 


\end{tikzpicture} 
\end{minipage}

%% \caption{\emph{Left:} \textbf{BAD} Adding sequential reductions like this, reintroduces memory coalescing issues. Consecutive threads nolonger access consecutive memory locations.\newline
%% \emph{Right:} \textbf{GOOD} Using sequential reduction but maintaining coalescing} 

\label{fig:redSeq}
\end{figure} 
