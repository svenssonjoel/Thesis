

\subsection{Implementation} 
EmbArBB is a deeply embedded language. There is a constructor for 
every operation that ArBB can perform. A deep embedding is useful when 
there is need to apply optimisations or transformations to the AST before 
executing the operations. However, EmbArBB currently relies entirely on 
ArBB to perform optimisations to the program. The only optimisation 
performed on the EmbArBB side is sharing detection. Detecting sharing 
reduces the number of calls into the ArBB C library. Should we add a 
GPU backend to EmbArBB, further GPU specific optimisations of the AST 
will likely become necessary. 
 

\subsubsection{Vectors} 
EmbArBB has support for one, two and three dimensional dense vectors,
represented by a datatype called {\tt DVector}. 

\begin{verbatim}
import qualified Data.Vector.Storable as V

data DVector d a 
  = Vector {vectorData  :: V.Vector a, 
            vectorShape :: d} 
\end{verbatim}  

The payload data in a {\tt DVector} is stored in a vector from the 
{\tt Data.Vector} library. There is also a {\tt d} parameter that specifies 
the shape of the DVector. 

%The {\tt DVector} type has two parameters. The first parameter {\tt d} 
%is a type level representation of the dimensionality of a vector. 
%The dimensionality is encoded using a datatype: 

The {\tt d} parameter is used for a type level representation of the 
dimensionality of a vector, as in the Repa library~\cite{REPA}. The 
dimensionality is encoded using the following types together with 
the {\tt Int} type. 

\begin{verbatim} 
data a :. b = a :. b  
infixl :. 

data Z = Z 
\end{verbatim} 

%There are also easy to use synonyms for the allowed cases:
The dimensions zero to three are represented as follows:

%% MS : Shouldn't there be a ref. to Repa (or to whatever paper first
%% used these kinds of types for arrays) here??  I put a suggestion
%% but it should be changed to the correct ref.

\begin{verbatim} 
type Dim0 = Z             
type Dim1 = Dim0 :. Int
type Dim2 = Dim1 :. Int 
type Dim3 = Dim2 :. Int
\end{verbatim} 

For example in the reduction functions provided by EmbArBB the 
output vector is of a dimensionality one less than the input vector 
used. Below is the type of {\tt addReduce} to illustrate this. 

\begin{verbatim} 
addReduce :: Num a 
           => Exp USize 
           -> Exp (DVector (t:.Int) a) 
           -> Exp (DVector t a)
\end{verbatim} 

A somewhat unfortunate side effect of this is that currently 
EmbArBB has two kinds of scalars,
{\tt Exp (DVector Dim0 a)} and {\tt Exp a}. The result of reducing 
a one-dimensional vector is a zero-dimensional vector. An operation 
called {\tt index0} converts from zero-dimensional vectors 
to scalars. 

Irregular container, in ArBB called Nested vectors, are represented in 
EmbArBB by the type {\tt NVector}. Currently there are no versions of the 
{\tt copyIn}, {\tt copyOut} or {\tt new} functions for nested vectors. 
The programmer must transfer dense vectors into the ArBB heap and then 
apply nesting to them as part of the computation to perform thereupon. This 
means that having a concrete representation for a NVector is currently 
not useful. In the hope of being able to implement some of the transfer 
functions, even without direct support from the ArBB VM, we represent 
NVectors as a vector of dense data together with a vector containing 
segment lengths.  

\begin{verbatim} 
data NVector a = 
       NVector { nVectorData     :: V.Vector a
               , nVectorNesting  :: V.Vector USize}
\end{verbatim} 

As part of the interface between Haskell and EmbArBB there are also mutable 
vectors, of a type called {\tt BEDVector}. These are vectors that reside in 
the ArBB heap and are represented only by an integer identifier with which 
the actual data can be accessed from ArBB. These are used to store the inputs 
and outputs of calls to {\tt execute} on a captured function. 
 
\begin{verbatim}
data BEDVector d a = 
       BEDVector { beDVectorID :: Integer
                  , beDVectorShape :: d}  
\end{verbatim}

New {\tt BEVector}s are created using the function {\tt new}.
The function {\tt copyIn} copies a {\tt DVector} from Haskell to the ArBB heap 
and there is a function {\tt copyOut} to retrieve data from ArBB.  


\subsubsection{The language} 
The EmbArBB language is implemented as a set of library functions, operating on an expression datatype:

\begin{verbatim} 
data Expr = Lit Literal
          | Var Variable 
            
          | Index0 Expr 
          | ResIndex Expr Int 
            
          | Call (R GenRecord) [Expr]  
          | Map  (R GenRecord) [Expr]   
            
          | While ([Expr] -> Expr)  
                  ([Expr] -> [Expr])  
                  [Expr] 
                             
          | If Expr Expr Expr 
          | Op Op [Expr]   
\end{verbatim} 
Most of the ArBB functionality is taken care of by the {\tt Op} constructor. 
The datatype {\tt Op} used to represent operations has over 120 
constructors, so only a selection is shown in figure~\ref{fig:OPS}. 
Having all these 120+ operations taken care of by one constructor in 
the expression datatype simplifies the implementation of the backend, 
since all of these operations are handled in a very similar way. The few 
remaining special ArBB capabilities such as loops, and function mapping 
and calling are handled by their own cases, discussed below. There are 
also some implementation specific details that result in their own 
constructors in the {\tt Expr} type, namely {\tt Index0} for turning a 
zero dimensional DVector into a scalar, and {\tt resIndex} that helps with 
implementing operations that have more than one result. An example of an 
such an operation is {\tt SortRank}, which returns both the sorted result 
of an input vector and a vector of indices that specifies a permutation 
that would have sorted the input vector. 

The {\tt Expr} type also has constructors for the {\em call} and {\em map} 
functionality. To call a function means to apply it to input data. The 
{\tt map} operation specifies elementwise application of a function over 
vectors, like NESL's {\em apply-to-each}~\cite{NESL}, so it corresponds 
to Haskell's {\tt map} and {\tt zipWith}. Both {\tt Call} and {\tt Map} 
take a {\tt(R GenRecord)}. This is inspired by the delayed expressions 
that enable the implementation of {\tt vapply} in Nikola~\cite{NIKOLA}. 
The {\tt R} is the reification monad used to create DAGs (directed acyclic 
graphs) from embedded language functions. These DAGs are part of a 
{\tt GenRecord} that contains all information that the ArBB code generator 
needs to generate the ArBB function. 

The {\tt While} loop is represented using higher order abstract syntax, 
that is using functions to represent the condition and body. The state 
is represented by a list of expressions; this needs to be generalised 
somewhat in order to support loops with general tuples in the state. 
Something more structured than a list is needed for this. 

The expression data type used in EmbArBB is untyped (not using a GADT). 
A typed interface to the language is supplied using the same phantom types 
method as used in {\em pan}~\cite{COMPILEEDSL} and many other Haskell embedded 
languages since then. The choice to use an untyped {\tt Expr} datatype was 
based on the wish to keep the backend (ArBB code generation) as simple as 
possible. 

%Currently there 
%are very few {\tt expr} to {\tt expr} transformations taking place before the code 
%generation so the benefits of having the typed representation would be small.
% ** Have commented this out for now as it may cause confusion, esp if
% the implementation changes before the workshop

\begin{verbatim}
-- Phantom types
type Exp a = E Expr 
\end{verbatim}  

As an example, the operation {\tt addReduce}, which reduces a vector across a specified dimension, is implemented as follows in the EmbArBB library:

\begin{verbatim}
addReduce :: Num a 
           => Exp USize 
           -> Exp (DVector (t:.Int) a) 
           -> Exp (DVector t a) 
addReduce (E lev) (E vec) = 
  E $ Op AddReduce [vec,lev]
\end{verbatim}  

%The {\tt addReduce} function takes a {\tt DVector} of any dimensionality greater than 0,
%thats what the (():.t) in its type specifies, and a integer ({\tt USize}) that specifies 
%the level to reduce across. The result is a {\tt DVector} of dimensionality one less 
%than the input vector. 



\begin{figure} 
\begin{minipage}{\linewidth}
\begin{Verbatim}[frame=single]
data Op =  
   -- elementwise and scalar
     Add | Sub | Mul | Div | Max | Min 
   | Sin | Cos | Exp 
   ... 
   
   -- operations on vectors 
   | Gather | Scatter | Shuffle | Unshuffle
   | RepeatRow | RepeatCol | RepeatPage 
   | Rotate | Reverse | Length | Sort 
   | AddReduce | AddScan | AddMerge
   ... 
\end{Verbatim}
\end{minipage}
\caption{  ArBB scalar, elementwise and vector operations, which are handled by the {\tt Op} constructor in the {\tt Expr} datatype.
          This is just a selection from the more than 120 different operations ArBB provides.}
\label{fig:OPS}
\end{figure}

\FloatBarrier

\subsubsection{Interfacing with Haskell and Code generation} 

The interface between ArBB and Haskell consists of the {\tt DVector}, {\tt NVector}, {\tt BEDVector} and {\tt BEScalar} types, 
the {\tt capture}, and {\tt execute} functions, and the ArBB monad with its {\tt withArBB} 
``run''-function. This section describes what happens when the programmer captures 
an embedded language function, and when {\tt execute} is called on a captured
function. 

Capture and execution of functions takes place in the {\tt ArBB} monad, which
manages state of type {\tt ArBBState}:

\begin{verbatim}
data ArBBState = 
  ArBBState 
    { arbbFunMap :: Map.Map Integer 
                            ArBBFun        
    , arbbVarMap :: Map.Map Integer 
                            VM.Variable 
    , arbbUnique :: Integer } 

type ArBBFun = (VM.ConvFunction, [Type], [Type])
\end{verbatim}

This state contains a map from function names to ArBB functions and their input and 
output types. The {\tt VM.ConvFunction} is how ArBB functions are represented by 
the ArBB-VM bindings. There is a map from vector and scalar IDs to their 
corresponding ArBB variables. The last item in {\tt ArBBState} is an Integer 
that is used to generate new function names and variable IDs as the programmer 
captures more functions or creates new arrays on the ArBB heap. 
Now, the {\tt ArBB} monad is defined as: 

\begin{verbatim}
type ArBB a = StateT ArBBState VM.EmitArbb a  
\end{verbatim}

{\tt VM.EmitArbb} is also a concept from the virtual machine bindings. It manages 
low-level functions (the {\tt VM.ConvFunction} functions) and implements an interface to 
the low level ArBB-VM API. 
{\tt EmitArBB} is the ArBB code generating monad from the arbb-vm bindings.
For more or less everything that can be done with the arbb-vm bindings, there is a function
of the form 
\begin{verbatim}
f :: arg1 -> ... argn -> EmitArBB out
\end{verbatim}
For example, for generating an operation node (such as +) in the ArBB IR, there is a
function of type
\begin{verbatim}
op_ :: Opcode 
     -> [Variable] 
     -> [Variable] 
     -> EmitArbb ()
\end{verbatim}
\noindent
while the function for generating a while loop in the ArBB IR has type
\begin{verbatim}
while_ :: (EmitArbb Variable) 
        -> EmitArbb a 
        -> EmitArbb a
\end{verbatim}
The details of the translation are omitted for brevity, as the approach is
standard.

When a function {\tt f} of type 
\begin{verbatim}
Exp tin1 -> ... -> Exp tinN -> Exp tout
\end{verbatim}
is captured, it is 
first applied to expressions that represent variable names. For each of the 
inputs, {\tt (tin1,..,tinN}), a variable is created. The result is an expression 
(or expressions) representing the function {\tt f}. On this expression, sharing 
detection is performed, and a directed acyclic graph (DAG) is created. 
The method of sharing detection used is based on the StableNames method~\cite{Gill}. 

Then, the code generation is implemented using a very 
direct approach; no extra optimisations or transformations are applied. 
This is a reasonable choice, since the whole point of ArBB is that the 
built-in JIT compiler knows and performs architecture specific optimisations. 
Sharing detection on the Haskell side makes sense because code generation for 
each node in the DAG results in at least two calls into the ArBB-VM API, which 
means going through the FFI and incurring the associated cost. Detecting the
sharing already in the host language should give a smaller workload for the 
ArBB JIT compiler, thus reducing the time spent on JITing.
It remains to be seen how important JIT cost will be in practice, however,
as we expect it to be amortised over a large number of executions of 
the JITed code. We will need to conduct experiments with a suite of larger 
examples in order to decide if sharing detection on the Haskell side is 
worthwhile.

%Now, in a system such as this the time it 
%takes to JIT is of lesser importance, the JIT cost will likely be amortised over 
%a large number of executions of the JITed code. 

Applying {\tt capture} to a function gives an object of type

\begin{verbatim}
type FunctionID = Integer

data Function i o = Function FunctionID
\end{verbatim}

The {\tt i} and {\tt o} parameters to {\tt Function} represent the input and output 
types of the captured function. As an example, capturing 
\begin{verbatim}
f :: Exp (DVector Dim1 Word32) 
   -> Exp (DVector Dim1 Word32)
\end{verbatim}
results in an object of type
\begin{verbatim}
Function (BEDVector Dim1 Word32) 
         (BEDVector Dim1 Word32)
\end{verbatim}
This is just phantom types placed over a function name that is just a {\tt String},
but it does offer a typed interface for the {\tt capture} and {\tt execute} functions.

The {\tt execute} function that launches a captured function takes a {\tt Function i o} 
object, inputs of type {\tt i} and outputs of type {\tt o}.
The function name is looked up in the {\tt ArBB} environment (the monad). The inputs
and outputs are also looked up in the environment and then the function 
is executed. 

\begin{verbatim} 
execute :: (VariableList a, VariableList b) 
         => Function a b -> a -> b -> ArBB ()       
execute (Function fid) a b = 
  do 
    (ArBBState mf mv _) <- S.get 
    case Map.lookup fid  mf of 
      Nothing -> error "execute: Invalid function" 
      (Just (f,tins,touts)) -> 
        do 
          ins <- vlist a 
          outs <- vlist b
          
          liftVM$ VM.execute_ f outs ins 
\end{verbatim} 

%The {\tt arbbUp}, {\tt arbbDown} and {\tt arbbAlloc} functions are provided by the 
%{\tt ArBBin} and {\tt ArBBOut} classes. Output storage needs to be allocated before the
%function is executed. The {\tt arbbAlloc b} uses the sizes and shapes of the output 
%to allocate proper storage for the results in ArBB memory.

The {\tt vlist} function goes through the heterogeneous list of inputs or outputs 
and looks up each of the elements in the {\tt arbbVarMap}; the result is a Haskell 
list of {\tt VM.Variable}. The function {\tt VM.execute\_} is part of the virtual 
machine API bindings, and corresponds directly to a C function in that library. 


%EmbArBB is built on top of the Haskell ArBB-VM bindings~


%\begin{figure}
%\begin{center}
%\includegraphics[width=5cm]{./img/stack}
%\end{center}
%\caption{EmbArBB is implemented on top of Haskell bindings to the ArBB virtual machine C API}
%\label{fig:stack}
%\end{figure}


\FloatBarrier
%\subsection{ArBB Code generation}
%Describe what happens on ``Capture'' and ``Execute'' 

%\subsection{summary} 
