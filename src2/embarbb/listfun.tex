% -- Get the number of columns
% getNCols :: Exp (DVector (t:.Int:.Int) a) 
%           -> Exp USize 



\begin{figure}
\begin{small}
\begin{Verbatim}
-- Scatter values into a vector 
-- resolves collisions by adding elements
addMerge :: Exp (DVector (t:.Int) USize) -- indices
         -> Exp USize                    -- res length
         -> Exp (DVector (t:.Int) a)     -- src
         -> Exp (DVector (t:.Int) a) 

-- Reduce a vector using addition
addReduce :: Num a 
          => Exp USize   -- rows, cols or pages
          -> Exp (DVector (t:.Int) a) 
          -> Exp (DVector t a) 

-- Segmented version of addReduce
addReduceSeg :: Num a 
             => Exp (NVector a)  -- nested input
             -> Exp (DVector Dim1 a)

-- Create a nested vector from a dense
applyNesting :: Exp USize  -- lengths or offsets
             -> Exp (DVector Dim1 USize) -- nesting
             -> Exp (DVector Dim1 a)     
             -> Exp (NVector a)


-- Create a vector with a constant value
constVector :: Exp USize  -- length
            -> Exp a      -- value
            -> Exp (DVector Dim1 a) 

-- Extract a column from a 2D vector
extractCol :: Exp USize              -- col. index
           -> Exp (DVector Dim2 a)  
           -> Exp (Vector a) 
 
-- Fill a portion of a vector with a constant 
fill :: Exp a                -- fill value 
     -> Exp USize            -- start 
     -> Exp USize            -- end 
     -> Exp (DVector Dim1 a) -- dst
     -> Exp (DVector Dim1 a) 

-- Gather elements from a vectors 
gather1D :: Exp (DVector Dim1 USize)  -- indices
         -> Exp a                     -- default
         -> Exp (DVector Dim1 a)      -- values
         -> Exp (DVector Dim1 a) 

-- Get the number of rows 
getNRows :: Exp (DVector (t:.Int:.Int) a) 
         -> Exp USize 

-- Turn a zero dimensional vector into a scalar 
index0 :: Exp (DVector Z a) -> Exp a 

-- Create a 2D vector by repeating a 1D vector
repeatRow :: Exp USize            -- #repetitions
          -> Exp (DVector Dim1 a) -- row
          -> Exp (DVector Dim2 a) 

-- Replace one column in a 2D vector 
replaceCol :: Exp USize            -- col
           -> Exp (DVector Dim1 a) -- new values
           -> Exp (DVector Dim2 a) 
           -> Exp (DVector Dim2 a) 

-- Convert every element of a vector to USize
vecToUSize :: Exp (Vector a) 
           -> Exp (Vector USize)

\end{Verbatim}
\end{small}
\caption{A list of EmbArBB functions that are used in the examples with 
         short descriptions} 
\label{fig:listoffun}
\end{figure} 

\begin{figure}
\fbox{
\parbox{8cm}{
\begin{itemize}
\item {\tt ISize} is an integer type used to specify for example offsets.
\item {\tt USize} is an unsigned integer type used for lengths or indices.
\item {\tt Boolean} replaces the {\tt Bool} type for EmbArBB programs. The 
reason for this is that ArBB internally represents booleans as 8bit words.
while the {\tt Storable} instance for {\tt Bool} does not. 
\item {\tt DVector} is the type of regular shaped vectors.
\item {\tt NVector} is the type of irregularly shaped vectors.
\end{itemize}
}
}
\caption{A list of EmbArBB types with short descriptions} 
\label{fig:listoftyp}
\end{figure} 


