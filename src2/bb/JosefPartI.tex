\section{Introduction}

Suppose you have just bought yourself a little programmable toy robot
and you want to design a small DSL in Haskell to program it. The robot
can turn, move forward and has a sensor to detect whether it is facing
an obstacle or not. Inspired by the robot language in \cite{} you come
up with a set of primitives shown in figure
\ref{fig:interface}. In particular, you would like to use the
do-notation to sequence programs.

\begin{figure} 
\begin{itemize} 
  \item \verb!move      :: Program ()! 
  \item \verb!turnLeft  :: Program ()!
  \item \verb!turnRight :: Program ()!
  \item \verb!sensor    :: Program Bool!
  \item \verb!cond      :: Program Bool -> Program () -> Program () -> Program ()!
  \item \verb!while     :: Program Bool -> Program () -> Program ()!
  \item \verb!instance Monad Program!
\end{itemize} 
\label{fig:interface} 
\caption{Proposed set of basic robot operations} 
\end{figure}

\emph{Example programs}


The next task is to implement this language. Compiling EDSLs like this
to something which the physical robot can execute is well know and
there are off-the-shelf techniques to use. But there is one quirk; how
does one generate an abstract syntax tree from a type which implements
the Monad interface?

Since you want to compile the language there must be an abstract
syntax tree representation of some sort of the language. The natural
thing is just to create a new type where each language construct has
its own constructor. But should you deal with the monadic constructs?
Well, the simplest solution would be to just add them to the type as well. The resulting type 

\begin{figure}
\begin{verbatim}
data BoolE = Lit Bool
         | Var String
         | (:||:) BoolE BoolE
         | (:&&:)  BoolE BoolE
         | Not BoolE  

data Program a where
  Move      :: Program ()
  TurnRight :: Program ()
  TurnLeft  :: Program ()

  Sensor    :: Program BoolE
  Cond      :: BoolE -> Program () -> Program () -> Program ()
  While     :: Program BoolE -> Program () -> Program ()
   
  Return    :: a -> Program a 
  Bind      :: Program a -> (a -> Program b) -> Program b
\end{verbatim}
\label{fig:program}
\caption{A data type for programs}
\end{figure}

\emph{More text}

In the rest of this paper we will see that this na\"ive method
actually can be made to work and that it's a practical, compositional
technique for reifying monads.
