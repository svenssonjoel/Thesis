\subsection{GPUs and CUDA} 
%\FloatBarrier
% This section provides an overview of GPU programming and architecture. 

The GPUs we target in this paper and with Obsidian are NVIDIA GPUs which
support CUDA. Here we provide some background information on GPUs 
that we hope will aid in the understanding of example programs throughout this paper. 
%% This section briefly outlines features of GPUs that have 
%% influenced the current version of Obsidian and that we make use of when 
%% implementing the algorithms in following sections. 


GPUs supporting CUDA are based on a scalable design. The GPU consists of a 
number of so-called multiprocessors (MPs) each consisting of a number of 
processing elements (PEs); sometimes these are referred to as cores or as 
stream processors. An MP also contains a local shared memory through which 
threads running on the PEs can communicate. The key that 
makes the architecture scalable is that a GPU consists of one or more 
of these MPs. This also influences the programming model significantly; since
threads can only communicate in a synchronised manner using the local memory 
(or in a very limited way using atomic operations) the programmer must 
partition the computation in such a way that its communication patterns 
follow this architectural constraint. The GPU also has access to a larger 
memory called the {\em global} or {\em device} memory. The sizes of these 
memories and the number of MPs and PEs per MP vary slightly between 
generations of GPUs. For example, the GPU used in the benchmarks 
(section \ref{sec:CSORTBenchmarks}) has a total of 1344 single-precision floating
point CUDA cores and has 2 GB global memory. The very latest GPU architectures 
from NVIDIA slightly digress from what is outlined here but mostly still 
adhere to the same programming model. For exact details refer to \cite{KEPLER}.  

The programming model that CUDA exposes is called single program multiple 
threads (SPMT) and is a slight variation of the SPMD concept. This programming 
model reflects the GPU architecture. In CUDA, the programmer writes a single 
program that is executed by many threads. Because of the scalable architecture,
with its one or more MP, these threads are grouped into {\em blocks} 
(now up to 1024 threads per block). A block of threads is executed on an MP; 
thus only threads within that block can communicate using the shared memory. 
The same CUDA program can be executed on any GPU along the scale, from the 
smallest with only one MP to the largest. The only difference is in the number 
of blocks of threads that can be executed in parallel. This constraint implies 
that all blocks must be free of any sequential dependencies. The collection of 
all blocks is called a {\em Grid}.

A CUDA program has two parts; there is a coordination program that runs on the 
CPU and there are kernels that execute on the GPU MPs. The program describing 
a kernel is parameterised over a blockId and a threadId so that 
decisions can be made from that information during execution. A typical 
CUDA program starts out by loading data into local memory using some 
function of blockId and threadId. It then computes on the local memory 
and uses a syncthreads primitive when exchanging values between threads. 
When the local computation is done the final results are written back into 
global memory again using some function of blockId and threadId. 
%A detail 
%here is that within a block, a smaller number of threads (32 of them) 
%called a {\em Warp} can communicate safely without using the synchronisation 
%primitive. This comes from how the GPU executes the warps; this is done 
%in SIMD fashion, all threads progress at the same pace (essentially that 
%they share a single instruction pointer). 

For more CUDA and GPU specifics see references \cite{KEPLER,CUDA}. 

\begin{figure}
\begin{small}
\begin{verbatim}
__global__ void addv(float *i0, 
                     float *i1, 
                     float *r){ 
  unsigned int ix = 
     blockIdx.x * blockDim.x + threadIdx.x;

  extern __shared__ float sm[]; 
  sm[threadIdx.x] = i0[ix] + i1[ix]; 
  r[ix] = sm[threadIdx.x];
} 
\end{verbatim}
\end{small}
\caption{The code illustrates elementwise addition of vectors. Shared memory 
(the sm array) is used to allow the {\tt addv} program to be run with the 
result {\tt r} pointing to the same memory area as either of the inputs.} 
%But mostly shared memory is used to illustrate the syntax involved.}
\label{fig:cudaKernel1}
\end{figure}

\begin{figure}
\begin{small}
\begin{verbatim}
#define BLOCK_SIZE 32
#define BLOCKS 4 
#define N (BLOCKS * BLOCK_SIZE)

int main(int argc, char **argv){
  float *v1, *v2, *r;
  float *dv1, *dv2, *dr;
  
  v1 = (float*)malloc(N*sizeof(float));
  v2 = (float*)malloc(N*sizeof(float)); 
  r = (float*)malloc(N*sizeof(float));

  //Generate or read input data.
  ... 

  //Allocate arrays in Global memory
  cudaMalloc((void**)&dv1, sizeof(float) * N ); 
  cudaMalloc((void**)&dv2, sizeof(float) * N ); 
  cudaMalloc((void**)&dr, sizeof(float) * N ); 
  
  //Copy data into Global memory
  cudaMemcpy(dv1, v1, sizeof(float) * N, 
                      cudaMemcpyHostToDevice);
  cudaMemcpy(dv2, v2, sizeof(float) * N, 
                      cudaMemcpyHostToDevice);
  
  //Launch the vector add kernel.
  addv<<<BLOCKS, BLOCK_SIZE, BLOCK_SIZE * sizeof(float)>>>
      (dv1,dv2,dr);
  
  //Launch further kernels on the data.
  ... 
  
  cudaMemcpy(r, dr, sizeof(float) * N , 
                    cudaMemcpyDeviceToHost);
  
  cudaFree(dv1);
  cudaFree(dv2);
  cudaFree(dr);
  
  //Show or further process results on the CPU. 
  ... 
}

\end{verbatim}
\end{small} 
\caption{This code starts a CUDA kernel on  
the GPU. The syntax $<<<${\tt nb,nt,sm}$>>>$ is used to set up a kernel
launch configuration. The {\tt nb}, {\tt nt} and {\tt sm} quantities refer to 
the number of blocks, the number of threads/block and the amount of shared 
memory.}
\label{fig:cudaCoord}
\end{figure}



%\FloatBarrier
