
\subsection {Discussion} 

It is instructive to compare the code of the two variants of counting
sort. The standard algorithm requires mutations and its parallel
implementation is rather complicated. The occurrence sort variation
avoids key synchronisations, making the algorithm
data-parallel and this gives it a performance advantage. However, in
applying the trick to remove synchronisation we have changed the
semantics of the function. In this case, it is still useful, albeit less
generally applicable. The idea of removing synchronisation is a
general and important idea for speeding up algorithms, but it is not
clear how general the method used in this paper is.  We currently do
not know of any other algorithms which could benefit from a similar
trick.

If the desired effect is to sort and remove duplicates 
(for an example of this see \cite{REMOVEDUPS}) then 
there is a clear benefit to using occurrence sort
over the {\tt sort} combined with {\tt unique} method. 

In this paper, we are entirely focused on the implementation and
performance of kernels. Reference \cite{CSORT} goes into the details
of performing data transfers in parallel with ongoing computations. We
have not addressed this issue at all in our implementations. This is
also the reason why we haven't performed any benchmarks against the algorithm in
\cite{CSORT}, because it would compare very different things. As a
piece of future work, it would be interesting to add the capability of
streaming data from the main memory to the GPU to our algorithms.

In counting sort, an auxiliary array holding the count of how many times 
each element occurs in the input array, the histogram array, is created. 
The size of this array is 
based on the range of elements occurring in the input. This means that 
more memory is needed for larger ranges and may imply that counting sort 
is suitable only up to some point. A related algorithm, known as radix 
sort \cite{Knuth}, addresses this issue by using multiple counting sort 
like passes, one for each digit position and thus limits this auxiliary 
memory usage to an array of size related to the radix (number of unique 
digits) used in the representation of the values being sorted. During 
our exploration of the counting sort algorithm on a GPU, we were not much 
concerned by memory usage; all data and auxiliary 
arrays fit easily in the device memory. The charts (\ref{fig:chart1}, \ref{fig:chart3} and \ref{fig:chart2}) indicate that our implementation offers the most 
benefit in the ranges labelled as $T10$ to $T17$; these are in fact relatively 
small ranges (1024 to 131072 unique values). Our current hypothesis 
is that the decrease in benefit of counting sort from range $T18$ and onward
is due to more and more complicated memory access patterns in the 
reconstruction phase. However, more in-depth profiling is needed to 
understand where the decrease in benefit comes from.
%%  However, we postulate that 
%% the decrease in benefit following $T18$ and onward are due to more complicated
%% memory access patterns during reconstruction

Obsidian is work in progress but this paper shows a step forward in its
capabilities. New kinds of kernels that operate on global arrays can be 
implemented. We also added atomic operations to the language. However, 
when dealing with atomic operations, we express the Obsidian program 
at a very low level. 
%% The exact implementation of global arrays is not yet fully 
%% decided upon. For example, the global arrays shown in this paper does 
%% not have a length associated with them. The size of a global array is rather
%% decided as a multiple of the size the kernel operates on in the case of 
%% local computations. In the case of computations that do not use any local
%% memory the length of global arrays is more free. These sizes are set when 
%% launching the kernels. 
%% Sometimes we made use of very low-level capabilities of Obsidian; when dealing 
%%with atomic operations. 
One task for future work is to find some way to raise 
the level of abstraction in that area. Another future direction in exploring 
the counting sort algorithm is to consider further optimisations.
%% Another future direction in exploring 
%% the counting sort algorithm specifically is to see what other kind of  
%% optimisations we can perform. 
One possibility is to work with sequentiality 
(more work per thread). Continuing exploration in that direction could lead 
to further improvements of Obsidian when it comes to low-level control.  

In the current version of Obsidian, we need to write some CUDA 
code by hand in order to implement the algorithm we describe. 
In the future we want to be able to do all coding without leaving 
Haskell. 

%% There is already an example of using Obsidian as the code 
%% generating back end of such an EDSL. This EDSL is called PEGGY and 
%% is described in \cite{cole2012beauty}. 

%% This case study turned out an interesting exercise in very low-level 
%% programming in Obsidian. Having the ability to drop down to a CUDA-like 
%% language allows us to make use of atomic operations... 


 
