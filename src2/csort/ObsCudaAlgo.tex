
% insert better head of section here 
\subsection{Implementing counting sort in Obsidian} 
\label{sec:parallel}

%This section shows the implementations of the various kernels 
%involved in our two variants of counting sort. The prefix sum 
%computation that is used does not vary in the implementations and 
%is shown in section \ref{sec:sklansky}. 

%Two different kinds 
%of {\em histogram} and {\em reconstruct} kernels are developed. 
%The kernels involved in implementing the prefix sum used between the 
%histogram and reconstruct phases are shown in section \ref{sec:Obsidian}.

%This section outlines the implementation of the different kernels 
%required by the counting sort algorithm. The main parts are 
%computing the {\em histogram}, {\em prefix sums} and the {\em reconstruct} 
%phase. 

%The prefix sums computation consists of 2 different kernels while the 
%two other phases are performed by a single kernel. The prefix sums are computed
%in parallel following a scheme originally designed by 
%J. Sklansky \cite{Sklansky}.  

%Initially, we assumed that in order to tweak the performance of this algorithm 
%we would substitute various prefix sums networks in the place of the Sklansky 
%network and evaluate the performance difference. It may still be the case 
%that the parallel prefix network used is an important aspect in the 
%performance of this algorithm. But for many data sets the execution time 
%of the reconstruct kernel completely dominates. In this paper we use two 
%different versions of the same Sklansky network. One implemented using pull 
%arrays and the other utilising push arrays. 

%is it J. Sklansky? (whats the J for ?) 
%Lots of the above should move to discussion, right ? 

%\subsection{Counting sort}

%The missing kernels in order to implement counting sort is histogram,
%which counts occurrences of values in the input, and reconstruct, which
%outputs the final sorted array.

\subsubsection{Histogram} 

When computing the histogram, atomic operations are needed. Index $i$
in the histogram array is incremented for each occurrence of the value
$i$ in the input array, and if there are multiple occurrences of $i$
then multiple threads will try to increment, leading to a possible race.
CUDA atomic operations must be applied to data stored at an actual memory 
location. This does not fit very well into our setting with virtual arrays 
(arrays that do not necessarily represent data in memory). But in Obsidian 
it is possible to drop down to a low enough level of abstraction to still 
implement this function. However, we are searching for suitable higher level 
abstractions to apply here.

%\begin{small}
%\begin{Verbatim}[samepage=true] 
%histogram :: GlobPull (EInt32) -> Final (GProgram (GlobPull (EWord32)))
%histogram (GlobPull ixf) = Final $
%      do global <- Output $ Pointer (typeOf (undefined :: EWord32))
%         forAllT $ \gix ->
%           do AtomicOp global (int32ToWord32 (ixf gix))  AtomicInc
%              return ()
%         return (GlobPull (\i -> index global i))
%\end{Verbatim}
%\end{small}

\begin{small} 
\begin{Verbatim}[samepage=true] 
histogram :: GlobPull EInt32 
             -> GProgram ()
histogram gpull = do
  global <- Output $ Pointer Word32 
  forAllT $ \gix -> 
     atomicOp global 
              (int32ToWord32 (gpull ! gix)) 
              AtomicInc
\end{Verbatim} 
\end{small} 

Below, the code generated from this Obsidian program is shown.

\begin{small}
\begin{Verbatim}[samepage=true] 
__global__ void histogram(int32_t *input0,
                          uint32_t *output0){
  
  atomicInc(output0+(uint32_t)(input0[((bid*bDim)+tid)]),
            0xFFFFFFFF);
  
}
\end{Verbatim}
\end{small}


The generated code uses \verb!atomicInc! to increment a value in an
array. The function is conditional, incrementing only if the value in
memory is larger than the second argument. Since we always want to
increment, no matter the value in memory, we've supplied {\tt
  0xFFFFFFFF}, which is the largest possible value.

The Obsidian code and the CUDA code are very similar in size and complexity. 
However, an important advantage of the Obsidian code is that it composes better
than the CUDA code. If the input to {\tt histogram} was a {\tt
  GlobPull} array produced by some other Obsidian function, then the
two functions would be fused, typically generating much faster code
and not allocating memory for the fused array.

%\newpage
\subsubsection{Reconstruct}

The kernel for the reconstruct step uses one thread per element in the
input array. Each thread is implemented as a loop to write its
corresponding element as many times as it should occur in the output.

\begin{small}
\begin{Verbatim}[samepage=true] 
reconstruct :: GlobPull EWord32 
               -> GlobPush EInt32
reconstruct (GlobPull ixf) = GlobPush f
  where f k = 
    do forAllT $ \gix ->
         let startIx = ixf gix 
         in  SeqFor (ixf (gix + 1) - startIx) $ \ix ->
               k (word32ToInt32 gix) (ix + startIx)
\end{Verbatim}
\end{small}

The generated CUDA code for \verb!reconstruct! can be seen
below. Modulo syntax and some index manipulation, the code is very
similar to the Obsidian code.

\begin{small}
\begin{Verbatim}[samepage=true]
__global__ void reconstruct(uint32_t *input0,
                            int32_t *output0){
  
  for (int i1 =  0;
       i1 < (input0[(((bid*bDim)+tid)+1)]-
             input0[((bid*bDim)+tid)]);
       i1++)
  {
    output0[(i1+input0[((bid*bDim)+tid)])] = 
      (int32_t)(((bid*bDim)+tid));
    
  }
  
}
\end{Verbatim}
\end{small}

Again, the difference between Obsidian and CUDA might seem small, but
just as above, the code in Obsidian composes better.

\subsection{Implementing occurrence sort in Obsidian} 
\label{sec:occur}

Occurrence sort uses an {\tt occurs} kernel instead of a 
histogram. The result of the occurs computation is an array with a one 
at indices corresponding to values occurring in the input array. The 
reconstruction of the sorted (and duplicate free) array is also done 
slightly differently.

\subsubsection{Occurs}

The occurs kernel is implemented by using the {\tt scatterGlobal} 
function from the Obsidian library. This function takes two arrays, one of 
indices to write to and a second of elements to write. In this case, it 
is applied to the input array and an array containing all ones 
({\tt replicateG 1}). 

\begin{small}
\begin{Verbatim}[samepage=true] 
occurs :: GlobPull EInt32 -> GlobPush EWord32
occurs elems = 
  scatterGlobal (fmap int32ToWord32 elems) (replicateG 1)
\end{Verbatim}
\end{small} 

The CUDA code generated from this function is shown below. 

\begin{small}
\begin{Verbatim}[samepage=true] 
__global__ void occurs(int32_t *input0,
                       uint32_t *output0){
   
  output0[(uint32_t)(input0[((bid*bDim)+tid)])] = 1;
  
}
\end{Verbatim} 
\end{small} 

When running this code, it is possible that many threads write to the same 
target element. Since all are writing a one, no atomic operations are needed; 
the result will still be one. 

It is worth comparing the Obsidian code for the {\tt histogram} kernel
and the {\tt occurs} kernel. The {\tt histogram} kernel is highly
imperative in nature, and requires atomic operations to manage
synchronisation between threads. The {\tt occurs} kernel, on the other
hand, has a very straightforward data-parallel implementation. Not only
is it simpler, but as we will see in section \ref{sec:Benchmarks}, it
is also significantly faster.

\subsubsection{Reconstruct} 

The reconstruct kernel for the occurrence sort algorithm  is
almost identical to the kernel used for standard counting sort. The
only difference is that a conditional can be used instead of a loop. 
This can be seen as a slight optimisation; the previous version of reconstruct 
could still be used in its place. 

\begin{small}
\begin{Verbatim}[samepage=true]
reconstruct :: GlobPull EWord32 
               -> GlobPush EInt32
reconstruct (GlobPull ixf) = GlobPush f
  where f k = 
    do forAllT $ \gix ->
         let startIx = ixf gix 
         in  Cond ((ixf (gix + 1) - startIx) ==* 1) $
               k (word32ToInt32 gix) startIx
\end{Verbatim}
\end{small}

This variant of reconstruct results in the generated code below. 

\begin{small}
\begin{Verbatim}[samepage=true]
__global__ void reconstruct(uint32_t *input0,
                            int32_t *output0){
   
  if ((input0[(((bid*bDim)+tid)+1)]-
       input0[((bid*bDim)+tid)])==1){

  output0[input0[((bid*bDim)+tid)]] = 
      (int32_t)(((bid*bDim)+tid));
  
  }

}
\end{Verbatim}
\end{small}





%\subsection{Histogram computations} 
%
%There are two different variants of the histogram computation used 
%in our counting sort implementations. The first does not really create 
%a histogram but creates an array that indicates what elements occurred. A
%one in position $p$ indicates that $p$ was present in the input array. 
%
%There is also a full histogram used in the full version of counting sort. 
%Here if $n$ occurs at position $p$ in the result array it indicates that 
%$p$ occurs $n$ times in the input array. 

%\subsubsection{First histogram} 
%
%First we implement the histogram for the counting sort variant that removes
%duplicate elements. This function is not really a histogram, but rather 
%it places a one at each index that correspond to an element in the data array.
%The first histogram is implemented by using the {\tt scatterGlobal} 
%function from the Obsidian library. This function takes two arrays, one of 
%indices to write to and a second of elements to write. In this case it 
%is applied to the input array and an array containing all ones 
%({\tt replicateG 1}). 
%
%\begin{small}
%\begin{Verbatim}[samepage=true] 
%histogram :: GlobPull (EInt32) -> GlobPush (EWord32)
%histogram elems = 
%  scatterGlobal (fmap int32ToWord32 elems) (replicateG 1)
%\end{Verbatim}
%\end{small} 
%
%The CUDA code generated from this function is shown below. 
%
%\begin{small}
%\begin{Verbatim}[samepage=true] 
%__global__ void histogram(int32_t *input0,uint32_t *output0){
%   
%  output0[(uint32_t)(input0[((bid*bDim)+tid)])] = 1;
%  
%}
%\end{Verbatim} 
%\end{small} 
%
%When running this code it is possible that many threads write to the same 
%target element. Since all are writing a one no atomic operations are needed, 
%the result will still be one. 

%\subsubsection{Full histogram}
%
%When implementing the full histogram atomic operations are needed. In this 
%variant index $i$ in the histogram array is incremented for each occurrence 
%of the value $i$ in the input array. 
%CUDA atomic operations must be applied to data stored at an actual memory 
%location. This does not fit very well into our setting with virtual arrays 
%(arrays that do not necessarily represent data in memory). But in Obsidian 
%it is possible to drop down to low enough level of abstraction to still 
%implement this function. However, we are searching for suitable higher level 
%abstractions to apply here. 

%\begin{small}
%\begin{Verbatim}[samepage=true] 
%fullHistogram :: GlobPull (EInt32) -> Final (GProgram (GlobPull (EWord32)))
%fullHistogram (GlobPull ixf) = Final $
%      do global <- Output $ Pointer (typeOf (undefined :: EWord32))
%         forAllT $ \gix ->
%           do AtomicOp global (int32ToWord32 (ixf gix))  AtomicInc
%              return ()
%         return (GlobPull (\i -> index global i))
%\end{Verbatim}
%\end{small}
%
%\begin{small}
%\begin{Verbatim}[samepage=true] 
%__global__ void fullHistogram(int32_t *input0,uint32_t *output0){
%  
%  atomicInc(output0+(uint32_t)(input0[((bid*bDim)+tid)]),0xFFFFFFFF);
%  
%}
%\end{Verbatim}
%\end{small}
%
%
%The generated code uses atomicInc to increment a value in an array. 
%The second argument {\tt 0xFFFFFFFF} is a maximum required by CUDAs atomicInc, 
%the value is incremented if it is less than this value. To implement a count, 
%the maximum possible value is used as this argument. 


%\subsection{Array reconstruction} 
%
%After the histogram computation and the prefix sum that computes the array 
%we call the position arrays is done its time to reconstruct the sorted output. 
%
%\subsubsection{Reconstruct} 
%
%The first variant of reconstruct is for the duplicate removing counting sort. 
%It takes the original data array and the position array as input. If element 
%$i$ is in the data array it is written to the position that is pointed out
%by index $i$ in the position array. 
%
%\begin{small} 
%\begin{Verbatim}[samepage=true] 
%reconstruct :: GlobPull (EInt32) -> GlobPull (EWord32)
%               -> GlobPush (EInt32)
%reconstruct inp pos = scatterGlobal (ixMap f pos) inp  
%  where
%    f gix = fmap int32ToWord32 inp ! gix
%\end{Verbatim}
%\end{small} 
%
%
%\begin{small}
%\begin{Verbatim}[samepage=true] 
%__global__ void kernel(int32_t *input0,uint32_t *input1,int32_t *output0){
%   
%  output0[input1[(uint32_t)(input0[((bid*bDim)+tid)])]] = 
%   input0[((bid*bDim)+tid)];
%  
%}
%\end{Verbatim}
%\end{small}
%
%There is a way to implement an optimised version of reconstruct for 
%the counting sort that removes duplicates. This version does not take the 
%original data array as input. Instead the output data is recreated from the 
%position array alone.  
%
%\begin{small}
%\begin{Verbatim}[samepage=true]
%reconstruct' :: GlobPull (Exp Word32) -> GlobPush (Exp Int32)
%reconstruct' (GlobPull ixf) = GlobPush f
%  where f k = do forAllT $ \gix ->
%                   let startIx = ixf gix in
%                   Cond ((ixf (gix + 1) - startIx) ==* 1) $
%                      k (word32ToInt32 gix) startIx
%\end{Verbatim}
%\end{small}
%
%\begin{small}
%\begin{Verbatim}[samepage=true]
%__global__ void reconstruct_opt(uint32_t *input0,int32_t *output0){
%   
%  if ((input0[(((bid*bDim)+tid)+1)]-input0[((bid*bDim)+tid)])==1){
%
%  output0[input0[((bid*bDim)+tid)]] = (int32_t)(((bid*bDim)+tid));
%  
%  }
%
%}
%\end{Verbatim}
%\end{small}




%\begin{small}
%\begin{Verbatim}[samepage=true] 
%
%reconstruct :: Exp Word32 
%            -> GlobalArray Pull (Exp Word32) 
%            -> GlobalArray Pull (Exp Word32) 
%            -> Kernel (GlobalArray Push (Exp Word32))
%reconstruct l inp pos = return $ 
%  flip mkGlobalPushArray l $ \k ->
%    forAllGlobal (globLen inp) $ \i ->
%      k (pos ! (inp ! i),inp ! i)
%
%\end{Verbatim} 
%\end{small}


%\subsubsection{Full reconstruction} 

%\begin{small}
%\begin{Verbatim}[samepage=true] 
%fullReconstruct :: GlobPull (EWord32) -> GlobPush (EInt32)
%fullReconstruct (GlobPull ixf) = GlobPush f
%  where f k = do forAllT $ \gix ->
%                   let startIx = ixf gix in
%                   SeqFor (ixf (gix + 1) - startIx) $ \ix ->
%                      k (word32ToInt32 gix) (ix + startIx)
%\end{Verbatim}
%\end{small}
%
%\begin{small}
%\begin{Verbatim}[samepage=true]
%__global__ void kernel(uint32_t *input0,int32_t *output0){
%  
%  for (int i1 =  0;
%       i1 < (input0[(((bid*bDim)+tid)+1)]-input0[((bid*bDim)+tid)]);
%       i1++)
%  {
%    output0[(i1+input0[((bid*bDim)+tid)])] = (int32_t)(((bid*bDim)+tid));
%    
%  }
%  
%}
%\end{Verbatim}
%\end{small}



%\subsection{The CUDA framework MOVE TO BENCHMARK SECTION} 

%Given the generated code for all the kernels described previously it is now 
%possible to construct the complete algorithm in CUDA by simply providing the 
%skeleton that launches them. This involves allocation memory on the CUDA device 
%for all intermediate results (here intermediate means between the launches of two 
%kernels). Data needs to be copied to the device and then the kernels launched.
%for example, the kernel generated from the Obsidian histogram function can 
%be launched on the GPU by issuing: 

%\begin{small}
%\begin{Verbatim}[samepage=true] 
%histogram<<<nblocks, nthreads,0>>>(dvalues,0,histresult);
%\end{Verbatim} 
%\end{small}

%This indicates that {\tt histogram} should be started using {\tt nblocks}
%number of blocks of each {\tt nthreads} number of threads and no shared 
%memory. 
%In the current status of Obsidian this ``kernel coordination'' code needs 
%to be described in CUDA. 

%Definitely not room for the actual code here! (Maybe not even pseudocode) 


 
%The histogram kernel is operating on a large global input array. It looks 
%at each element in the input array and then puts a one in the output 
%array to signal the presence of a particular value. 

%This particular kernel does not need to use any local memory on the GPU and 
%there is no communication between threads. To implement this kernel it is 
%suitable to use Obsidian's global arrays.
 

%The kernel takes two inputs, a maximum length related to the size of the largest 
%element in the input data and a global array containing that data. A global 
%push array is created that contains a one at each index which corresponds to a
%value in the input data.

% \subsection{Computing prefix sums} 

%The prefix sum computation is different from the histogram one. Here threads
%needs to communicate and thus a kernel that utilises the GPU shared memory 
%is more suitable. 

%\begin{small}
%\begin{Verbatim}[samepage=true] 
%sklansky :: (Choice a, Syncable (Array Pull) a) 
%         =>Int
%         -> (a -> a -> a)
%         -> Array Pull a
%         -> Kernel (Array Pull a)
%sklansky 0 op = pure id
%sklansky n op = pure (twoK (n-1) (fan op)) 
%              ->- sync 
%              ->- sklansky (n-1) op 
%
%fan op arr = conc (a1, fmap (op c) a2) 
%    where 
%      (a1,a2) = halve arr
%      c = a1 ! (fromIntegral (len a1 - 1)) 
%\end{Verbatim} 
%\end{small}
%
%With this Obsidian code CUDA kernels can be generated that operate on certain 
%small sized sub arrays (called blocks). The factors that limit the number of 
%elements that the kernel can operate upon are size of shared memory and number 
%of threads per block that the CUDA device can launch. For this particular 
%kernel the number of threads limits it to operate on a maximum of 512 or 
%1024 elements depending on generation of GPU.  
%In order to build a large parallel prefix from these small ones an approach 
%similar to the one taken in \cite{LargeScan} is implemented. This requires 
%that the {\tt sklansky} kernel also outputs the maximum element from each 
%small block into a separate array of ``block maximums''. On the 
%block maximums a second prefix sum is computed and the result of that is then 
%distributed over the elements of the other blocks in the appropriate way. 
%So, to implement this the {\tt sklansky} function is modified into a function 
%that also outputs the block-maxima to a separate array: 
%
%\begin{small} 
%\begin{Verbatim}[samepage=true] 
%sklanskyM n op = sklansky n op ->- pure pairMax 
%  where
%    pairMax a = (a,
%                 let i = fromIntegral (len a - 1)
%                 in singleton (a ! i))

%\end{Verbatim}
%\end{small}

%The next step is the kernel that distributes (scanned) block-maximums. This is 
%again a kernel that does not need to use any local storage or communication
%between threads so the global array approach is suitable. The  inputs 
%to this kernel is, the block size, the array block maximums and the array 
%of elements we operate on. The output is an array where the element i of 
%the block maximums array is combined with each element in block i+1 of the 
%input data array.  

%\begin{small}
%\begin{Verbatim}[samepage=true] 
%distribute :: Exp Word32 
%           -> (GlobalArray Pull (Exp Word32), 
%               GlobalArray Pull (Exp Word32)) 
%           -> Kernel (GlobalArray Push (Exp Word32)) 
%distribute bsize (bmaxs,arr) = pure distr arr1
%    where 
%      distr arr = 
%          pushGlobal (zipWithGlobal (+) 
%                                    (expand bsize bmaxs) 
%                                    arr)
%      expand n arr =
%        mkGlobalPullArray (\ix -> arr ! (ix `div` n))  
%                          (globLen arr * n) 
%\end{Verbatim} 
%\end{small}

%Distribute works by expanding the array of block maximums, that is repeating each 
%of its elements a certain number of times and then performing a traditional 
%{\tt zipWith} operation with the values array. 

%The Obsidian code for {\tt sklansky} uses the {\tt conc} operation and the 
%generated code has an unfortunate conditional that is executed by each thread.
%Because of this another version was implemented using push arrays. The push 
%array version uses half as many threads and each thread computes both 
%branches of the conditional from the previous version. 

%\begin{small}
%\begin{Verbatim}[samepage=true] 
%ilv2v :: Choice a => Int -> (a -> a -> a) -> 
%                    Array Pull a -> Array Push a
%ilv2v i g arr 
%   = mkPushArray (\k -> a5 !* k *>* a6 !* k) n
%  where
%    n  = len arr
%    n2 = n `div` 2
%    a1 = resize (ixMap left arr) (n-n2) 
%    a2 = resize (ixMap right arr) (n2)  
%    a3 = resize (ixMap mid arr) (n2) 
%    a4 = zipWith g a3 a2
%    a5 = ixMap left (push a1)
%    a6 = ixMap right (push a4)
%    left = insertZero i
%    right = flipBit i  . left
%    mid = (.|. oneBits i) . left
%
%fan1 n op arr = ilv2v n op arr
%  
%%sklansky1 n op = [ilv2v i op | i <- [0..(n-1)]]
%\end{Verbatim} 
%\end{small}
%assumes we provide compose.
%maybe we should assume ilv2v is a library function 
%  and just explain what it does. 
%maybe Mary wants to tweak the above slightly ? 

%For space reasons we omit all examples of generated code. For many examples 
%of the kind of code Obsidian generates see references \cite{JSLIC,PUSH}. 
