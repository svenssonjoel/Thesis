\subsection{Introduction} 

Sorting is an ever important, ever fascinating field of computer
science. The introduction of GPUs has introduced new challenges in
designing fast sorting algorithms.

This paper focuses on counting sort together with a variation which
removes duplicate elements. We call this variation
{\em occurrence sort}. Counting sort is a non-comparing sort
suitable for parallel implementation. The occurrence sort variation 
presented here is interesting because it seems to be a particularly
good fit for executing on the GPU. It has a very natural functional,
data-parallel implementation. We believe we are the first to study
this variation in the literature.

These algorithms are explored in Obsidian \cite{JSLIC,PUSH}, a domain
specific language targeting GPU programming. The goal of Obsidian is
to strike a balance between high-level constructs and low-level control. 
We want to provide the programmer with a set of tools for low-level 
experimentation with the details that influence performance when 
implementing GPU kernels. The counting sort
case study shows that Obsidian generates competitive kernels with
relatively little programmer effort. That Obsidian is an embedded language 
allows us to rapidly experiment with the addition of features and with varying 
programming idioms. The version used in this case study
adds global arrays and atomic instructions to Obsidian, see section 
\ref{sec:OBSHist}. 

The contributions of this paper are:
\begin{itemize}
\item We provide measurements (section \ref{sec:Benchmarks}) showing
  that counting sort is a competitive algorithm for sorting keys on the GPU,
  outperforming the sorting implementation in the library
  Thrust\cite{THRUST} in many cases.
\item Occurrence sort is shown to be particularly
  suitable for implementing on the GPU (section \ref{sec:occur}) and
  performs well (section \ref{sec:Benchmarks}).
\item The Obsidian implementation of the two sorting algorithms is
  detailed along with the generated CUDA (sections
  \ref{sec:Obsidian} and \ref{sec:parallel}).
\end{itemize}


\subsubsection{Related work} 

Sorting has applications in the computer graphics field \cite{sintorn}. 
Example instances of sorting and duplicate element removal in computer graphics 
can be identified in references \cite{Olsson} and \cite{Karras}. Another example 
of where sorting and the removal of duplicates have applications is 
in databases \cite{REMOVEDUPS}. 

%% Sorting on the GPU is interesting both for computer graphics
%% applications \cite{SINTORN} and for using the GPU to offload the
%% processor, for example in database applications \cite{REMOVEDUPS}.

The paper \cite{CSORT} implements counting sort in CUDA, and also optimisations 
that overlap computations with transferring data to and from the GPU.
Our work differs by being implemented in a high-level embedded
language and also by implementation of the occurrence sort variant of counting sort. We have not been concerned with
trying to overlap computations and data transfer, but have instead
focused solely on computations within the GPU.

Obsidian is a high-level embedded language for GPU
programming. Other such approaches include Accelerate
\cite{Acc} and Nikola \cite{NIKOLA}. Accelerate and Nikola both take an even
higher level approach compared to Obsidian and abstract more from GPU
details. Another example of providing a high-level interface to
programming the GPU is the C++ library Thrust \cite{THRUST}.
%\cite{FELDSPAR} 

%% With the Obsidian language we raise the level of abstraction for 
%% GPU kernel implementation. In order to generate CUDA code that performs well, 
%% Obsidian tries to strike a balance between high level constructs and low level 
%% control. 

%% This paper describes a case study. We implement a sorting algorithm known 
%% as counting sort. As an extra bonus we explore a variant of counting sort 
%% that can be given a nice purely functional description and lends itself well 
%% to parallelisation. The experimental results in section \ref{sec:Benchmarks} 
%% compares the performance of the counting sort algorithm we implement here to 
%% the NVIDIA Thrust \cite{THRUST} library. The Obsidian generated kernels perform
%% very well at relatively light work effort. 

%% Counting sort places new demands on Obsidian and points out areas of 
%% improvement. Some left to do, some already implemented and used here. 
%% First, in order to implement counting sort we added atomic operations to 
%% Obsidian. However, currently you can only use them by very low level features 
%% of Obsidian. This is shown in section \ref{sec:parallel}. This paper also 
%% shows for the first time the new global arrays and what they can be used for.
   

%Vague intro ideas.
%\begin{itemize}
%  \item Quickly, what is our idea. 
%  \item Why is this sorting useful.
%        (Or is i just a fun experiment.) 
%  \item GPUs are pretty cool 
%  \item This is an Obsidian case-study. 
%  \item Sneak in ref to Feldspar.
%  
%\end{itemize} 

%\cite{CSORT} 
%\cite{FELDSPAR}
