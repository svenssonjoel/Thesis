

\section{Introduction} 

%% table 


\begin{figure}
\begin{center}
  \begin{tabular}{  l | r | r}
    %\hline
    Kernel               & Threads   & ms / grid   \\ \hline
    bitonic\_sort(CUDA)  & 512       & 30.28       \\      
    tsort1               & 512       & 15.82       \\
    tsort2               & 256       & 11.64       \\ 
    vsort1               & 512       & 14.96       \\ 
    vsort                & 256       &  9.33       \\ 
    %\hline
  \end{tabular}
\end{center}
\label{fig:expressive-table}
\caption{Table is showing the time it takes for a grid of 32678 blocks of 
threads to execute.Each grid was launched 1000 times, the total execution time 
was then divided by 1000 to get the listed figures. The GPU used is a NVIDIA GTX480} 
\end{figure}


%----------------------------------------------------------------------------

Graphics Processing Units (GPUs) are parallel computers with hundreds 
to thousands of processing elements. The CUDA and OpenCL languages  
make available the power of the GPU to programmers interested in general 
purpose computations. In CUDA and OpenCL, the programmer writes {\em kernels}, 
Single Program Multiple Data (SPMD) programs that are executed by groups 
of threads on the available processing elements of the GPU.  

CUDA and OpenCL are general purpose programming languages, mirroring the 
increased capabilities of a modern GPU to target that domain. However, these 
languages lack compositionality. 
Also, being based in C/C++ means that the core idea in a program may
not be easily visible.  

\subsection{Embedded DSLs for GPGPU programming} 
We are aiming for a GPU programming language that is more
concise than mainstream
languages such as CUDA and OpenCL. 
Obsidian is a
domain specific embedded language (DSEL) implemented in Haskell.
When an Obsidian program is run, a representation of the program is created 
as a syntax tree. For more information on EDSL implementation see~\cite{COMPILEEDSL}.
The program representation generated when running an Obsidian program 
is compiled into CUDA code. We are also working on an OpenCL backend.

Our approach is different from that of other Haskell DSELs targetting
GPUs~\cite{ACCELERATE,NIKOLA,BARRACUDA}. 
We do not try to abstract away from 
all the peculiarities of GPU programming, but rather provide a higher 
level language in which to experiment with them.
For instance,
Accelerate provides a standard set of basic operations such as 
{\tt map}, {\tt reduce} and {\tt zipWith} as built in skeletons, implemented
with the help of small, predefined, hand-tuned CUDA kernels~\cite{ACCELERATE}. 
Obsidian, on the other hand, allows the user to experiment with the
{\em generation} of small kernels for fixed size array inputs
from higher level descriptions.
It is intended to allow the user to play with the kinds of tradeoffs that are important
when writing such high performance building blocks; in this paper, the main consideration
is the number of array elements of the input and output that are manipulated
by a single thread in the generated CUDA code.
An important aspect of Obsidian is the symbolic array representation used, along
with its associated {\tt sync} operation. As we shall see, the {\tt sync} operation
allows the programmer to guide code generation and control parallelism and thread
use~\cite{JSLIC}.




%Today there are number of EDSLs written in Haskell targeting 
%GPUs~\cite{ACCELERATE,NIKOLA,BARRACUDA}. We justify our work on yet 
%another by having a slightly different approach. We do not try to abstract away from 
%all the peculiarities of GPU programming but rather giving a higher 
%level language in which to experiment with them.
%trying to appeal to a slightly different target audience. 
%We try to cater not for the advanced functional programmer who wants to 
%run his programs on GPUs, but rather to offer a higher level 
%language to the CUDA/OpenCL hacker. 
%In this language we want the user to be 
%able to quickly experiment with the kind of tradeoffs that are important 
%when writing a high performance GPU kernel.

%Unlike the other approaches that are complementary to our Haskell GPU EDSL, 
%Obsidian does not provide the usual set of basic operations such as 
%{\tt Map}, {\tt Reduce} and {\tt ZipWith} as built in skeletons. We 
%want a language where the programmer can experiment with implementing 
%his or her own versions of these basic operations. So far, we make
%this possible in Obsidian through a {\tt Sync} operation and the Array 
%representation we use. The {\tt Sync} operation, using target domain 
%nomenclature, is the programmers tool to guide code generation and 
%obtain parallelism. The presentation of Obsidian here will be somewhat 
%cursory, for a more complete view of the various versions and approaches 
%we have tried, see~\cite{JSLIC}.
%More details on this later.

In Obsidian, a kernel that sums an array can be expressed as: 
\begin{codesize}
\begin{verbatim}
sum :: Array IntE -> Kernel (Array IntE) 
sum arr | len arr == 1 = return arr
        | otherwise    = (pure (fmap (uncurry (+)) . pair) 
                          ->- sync 
                          ->- sum) arr                       
\end{verbatim}
\end{codesize}
The result of running this kernel on an eight element input array, 
{\tt runKernel sum (namedArray ``input'' 8)}, is an intermediate representation 
of the computation (shown in slightly pretty-printed form): 
\begin{codesize}
\begin{verbatim}
arr0 = malloc(16)
par i 4 {
arr0[i] = ( +  input[( *  i 2 )] input[( +  ( *  i 2 ) 1 )] );
}Sync
arr1 = malloc(8)
par i 2 {
arr1[i] = ( +  arr0[( *  i 2 )] arr0[( +  ( *  i 2 ) 1 )] );
}Sync
arr2 = malloc(4)
par i 1 {
arr2[i] = ( +  arr1[( *  i 2 )] arr1[( +  ( *  i 2 ) 1 )] );
}Sync
\end{verbatim}
\end{codesize}
The named intermediate arrays in this representation are then laid 
out in GPU shared memory and CUDA code can be generated (here for arrays of length eight)\footnote{An alignment qualifier for shared memory 
has been omitted to save space in the listings showing generated code}: 
%\pagebreak
\begin{codesize}
\begin{verbatim}
__global__ void sum(int *input0,int *result0){
  unsigned int tid = threadIdx.x;
  unsigned int bid = blockIdx.x;
  extern __shared__ unsigned char sbase[];
  (( int *)sbase)[tid] = 
    (input0[((bid*8)+(tid*2))]+
     input0[((bid*8)+((tid*2)+1))]);
  __syncthreads();
  if (tid<2){
    (( int *)(sbase + 16))[tid] = 
      ((( int *)sbase)[(tid*2)]+
       (( int *)sbase)[((tid*2)+1)]);
  }
  __syncthreads();
  if (tid<1){
    (( int *)sbase)[tid] = 
      ((( int *)(sbase+16))[(tid*2)]+
       (( int *)(sbase+16))[((tid*2)+1)]);
  }
  __syncthreads();
  if (tid<1){
    result0[(bid+tid)] = (( int *)sbase)[tid];
  }
}
\end{verbatim}
\end{codesize}
%The CUDA code above is generated by issuing the command:
%\begin{codesize}
%\begin{verbatim}
%genKernel ``sum'' sum (namedArray undefined 8 :: Array IntE)
%\end{verbatim}
%\end{codesize}
%\noindent
%and the kernel code is generated for arrays of length eight.  



%----------------------------------------------------------------------------
\subsection{Arrays in Obsidian}

An array is represented by an indexing function and a length:
\begin{codesize} 
\begin{verbatim}
data Array a = Array (UWordE -> a) Word32 
\end{verbatim}
\end{codesize}
This array representation has served us well. It has these properties:
\begin{itemize} 
\item Fusion of operations is automatic.
\item It naturally describes a data-parallel computation suitable for CUDA/OpenCL generation.
\item Many basic operations can be implemented: {\tt map}, {\tt zipWith} etc.
\end{itemize}
Using this array representation in a DSEL is not new; the first occurence that we know of is in
Pan~\cite{PAN}. Similar array representations have also later been 
used in Feldspar~\cite{FELDSPAR}, and more recently also in the Repa library~\cite{REPA}.
Functions for indexing and getting the length of arrays are as follows:
\begin{codesize} 
\begin{verbatim}
(!) :: Array a -> UWordE -> a 
(Array ixf _) ! ix = ixf ix 

len :: Array a -> Word32
len (Array _ n) = n 
\end{verbatim}
\end{codesize}
%Implementing a {\em Functor} instance for the {\tt Array} datatype is possible:
A {\em Functor} instance for the {\tt Array} datatype is
\begin{codesize} 
\begin{verbatim}
instance Functor Array where 
  fmap f arr = Array (\ix -> f (arr ! ix)) (len arr) 
\end{verbatim}
\end{codesize}
Now, composed applications of {\tt fmap} will be automatically fused. This is 
illustrated in the example program below and the CUDA generated
from it.  
\begin{codesize} 
\begin{verbatim}
mapFusion :: Array IntE -> Kernel (Array IntE) 
mapFusion = pure (fmap (+1) . fmap (*2)) 
\end{verbatim}
\end{codesize}

\begin{codesize} 
\begin{verbatim}
__global__ void mapFusion(int *input0,int *result0){
  unsigned int tid = threadIdx.x;
  unsigned int bid = blockIdx.x;
  
  result0[((bid*32)+tid)] = ((input0[((bid*32)+tid)]*2)+1);
}
\end{verbatim}
\end{codesize}
Both of these code listings need explanation. In the Haskell 
code, {\tt mapFusion} has type {\tt Array IntE -> Kernel (Array IntE)};
{\tt Kernel} is a state monad that accumulates CUDA code as well as provides 
new names for intermediate arrays. Neither of these features of the monad 
is activated by this example though. The {\tt pure} function is defined 
using the monad's {\tt return} as {\tt pure f a = return (f a)}. In this case, 
it lifts a function of type {\tt Array IntE -> Array IntE} into 
a kernel. 

The generated CUDA code computes the result array using a number of threads 
equal to the length of that array. In this case, the kernel was generated
to deal with arrays of length $32$. The important detail to notice in the 
CUDA code is that there is no intermediate array created between the 
{\tt (*2)} and the {\tt (+1)} operations. 

The {\tt mapFusion} example could just as well have been implemented using 
the kernel sequential composition combinator, {\tt ->-}. 
\begin{codesize} 
\begin{verbatim}
mapFusion :: Array IntE -> Kernel (Array IntE) 
mapFusion = pure (fmap (*2)) ->- pure (fmap (+1)) 
\end{verbatim}
\end{codesize}
Exactly the same CUDA code is then generated.

In some cases, it is necessary to force computation of 
intermediate arrays. This can be used to share partial computations between 
threads and to expose parallelism. In Obsidian, the tool for this
is called {\tt sync}, a built-in kernel. Using {\tt sync} as follows prevents fusion of the
two operations:
\begin{codesize} 
\begin{verbatim}
mapUnFused :: Array IntE -> Kernel (Array IntE) 
mapUnFused = pure (fmap (*2)) ->- sync ->- pure (fmap (+1))
\end{verbatim}
\end{codesize}
The generated CUDA code now stores an intermediate result in local shared
memory before moving on. 
\begin{codesize} 
\begin{verbatim}
__global__ void mapUnFused(int *input0,int *result0){
  unsigned int tid = threadIdx.x;
  unsigned int bid = blockIdx.x;
  extern __shared__ unsigned char sbase[];
  (( int *)sbase)[tid] = (input0[((bid*32)+tid)]*2);
  __syncthreads();
  result0[((bid*32)+tid)] = ((( int *)sbase)[tid]+1);
}
\end{verbatim}
\end{codesize}
Intermediate arrays are laid out in the {\tt sbase} array in shared memory. 
Since we may store arrays of many different types in the same locations 
of the shared memory at different times during the execution, the type casts
used in the code above are necessary. 

%----------------------------------------------------------------------------
\subsection{Sync and parallelism}

The {\tt sync} operation also enables the writing of parallel reduction 
kernels. A reduction operation is an operation that takes an array as input
and
produces a singleton array as output. 

First, we define {\tt zipWith} and {\tt halve} on Obsidian arrays.
\begin{codesize} 
\begin{verbatim}
zipWith :: (a -> b -> c) -> Array a -> Array b -> Array c
zipWith op a1 a2 = Array (\ix -> (a1 ! ix) `op` (a2 ! ix)) 
                   (min (len a1) (len a2))

splitAt :: Word32 -> Array a -> (Array a, Array a) 
splitAt n arr = 
  (Array (\ix -> arr ! ix) n , 
   Array (\ix -> arr ! (ix + fromIntegral n)) (len arr - n))

halve arr = splitAt ((len arr) `div` 2) arr
\end{verbatim}
\end{codesize}
A reduction kernel that takes an array whose length is a power of two 
and gives an array of length one can be defined recursively. 
Defining kernels recursively results in completely unrolled CUDA kernels,
and kernel input size must be known at compile time.
The approach to reduction taken here is to split the input array 
into two halves and then apply {\tt zipWith} of the combining function to the 
two halves, repeating the process until the length is one. 
\begin{codesize} 
\begin{verbatim}
reduceS :: (a -> a -> a) -> Array a -> Kernel (Array a) 
reduceS op arr | len arr == 1 = return arr
               | otherwise    = 
                 (pure ((uncurry (zipWith op)) . halve)
                  ->- reduceS op) arr
\end{verbatim}
\end{codesize}
%This Obsidian Kernel takes an array of length $2n$ as input and the output 
Since the output of this kernel is of length one, and the number 
of elements in the output array specifies the number of threads used to compute it, 
this function, {\tt reduceS}, defines a sequential reduction. 
The generated code for arrays of length eight is 
\pagebreak
%The code is generated with the 
%command {\tt genKernel ``reduceSAdd'' (reduce (+)) (namedArray undefined 8 :: Array IntE)}, 
%meaning that a CUDA reduce-with-addition kernel for arrays of length eight is generated. 
\begin{codesize} 
\begin{verbatim}
__global__ void reduceSAdd(int *input0,int *result0){
  unsigned int tid = threadIdx.x;
  unsigned int bid = blockIdx.x;
  
  result0[(bid+tid)] = 
    (((input0[((bid*8)+tid)]+
       input0[((bid*8)+(tid+4))])+
      (input0[((bid*8)+(tid+2))]+
       input0[((bid*8)+((tid+2)+4))]))+
     ((input0[((bid*8)+(tid+1))]+
       input0[((bid*8)+((tid+1)+4))])+
      (input0[((bid*8)+((tid+1)+2))]+
       input0[((bid*8)+(((tid+1)+2)+4))])));
}
\end{verbatim}
\end{codesize}
Sequential reduction is not very interesting for GPU execution, but 
the fix is simple. A well placed use of {\tt sync} indicates that 
we want to compute, after each {\tt zipWith} phase, the intermediate 
arrays using as many threads as that intermediate array is long. 
The effect is shown in the code below. 
\begin{codesize} 
\begin{verbatim}
reduce :: Syncable Array a 
          => (a -> a -> a) -> Array a -> Kernel (Array a)
reduce op arr | len arr == 1 = return arr
              | otherwise    = 
                (pure ((uncurry (zipWith op)) . halve)
                 ->- sync 
                 ->- reduce op) arr
\end{verbatim}
\end{codesize}
The CUDA code for reduction with addition on eight
elements is
\begin{codesize} 
\begin{verbatim}
__global__ void reduceAdd(int *input0,int *result0){
  unsigned int tid = threadIdx.x;
  unsigned int bid = blockIdx.x;
  extern __shared__ unsigned char sbase[];
  (( int *)sbase)[tid] = 
    (input0[((bid*8)+tid)]+
     input0[((bid*8)+(tid+4))]);
  __syncthreads();
  if (tid<2){
    (( int *)(sbase + 16))[tid] = 
      ((( int *)sbase)[tid]+
       (( int *)sbase)[(tid+2)]);   
  }
  __syncthreads();
  if (tid<1){
    (( int *)sbase)[tid] = 
      ((( int *)(sbase+16))[tid]+
       (( int *)(sbase+16))[(tid+1)]);    
  }
  __syncthreads();
  if (tid<1){
    result0[(bid+tid)] = (( int *)sbase)[tid];  
  }
}
\end{verbatim}
\end{codesize}           
In this generated CUDA, three phases can be identified. 
The first uses four threads to compute a four element intermediate array;
the second uses two threads, and so on. At the very end, a single thread
copies the result from local shared memory to global memory. 

\subsection{Drawbacks of Obsidian Arrays}
\label{sec:Drawbacks}

%*** THE STUFF BELOW IS IN CONFLICT WITH WHAT I ADDED ABOVE!
%When using Obsidian we have available to us a toolbox of higher order
%functions that are familiar to functional programmers. One example of such a 
%function is {\tt zipWith}. 

%\begin{codesize} 
%\begin{verbatim}
%example1 :: (Array IntE, Array IntE) 
%            -> Kernel (Array IntE)
%example1 (arr1,arr2) = return$ zipWith (+) arr1 arr2
%\end{verbatim}
%\end{codesize}

%The body of this function looks very familiar to Haskell programmers. The 
%type of the function however is a little bit different than what we are 
%used to from everyday Haskell usage. In Obsidian we have a concept called 
%a Kernel. Computations of types {\tt a -> Kernel b} can be turned into 
%runnable GPU code by providing an input. 
%*** VERY VAGUE AND ODD 
%*** TODO: Explain kernel ? 

%The following code listing shows the CUDA code we can generate from 
%the Obsidian program given above: 

% *** TODO: Needs a code size between tiny and small. 
 
%\begin{codesize}
%\begin{verbatim}
%__global__ void example1(int *input0,int *input1,int *result0){
%  unsigned int tid = threadIdx.x;
%  unsigned int bid = blockIdx.x;
  
%  result0[tid] = (input0[((bid*32)+tid)]+
%                  input1[((bid*32)+tid)]);
  
%}
%\end{verbatim}
%\end{codesize}

%The CUDA code above writes into a result array the pairwise sums from 
%its two input arrays, as expected. 
%*** There are many details here. tid, bid, 32. explain now or push to later?

% \subsection{Obsidian arrays} 

%In Obsidian an array is represented by an indexing function and a length. 
%\begin{codesize}
%\begin{verbatim}
%data Array a = Array (IxExp -> a) Word32
%\end{verbatim}
%\end{codesize}

%And {\tt IxExp} is defined as {\tt UWordE}, that is an expression 
%representing a unsigned 32bit value. The length of an array is represented
%with a unsigned 32bit value as well but here it has been beneficial 
%to use a built in Haskell type in order to be able to use this 
%information at program generation time rather than program run time. 


%This array representation specifies how to compute each index of an Arrays. 
%If we are interested in the value at location 5 in an array all we need 
%to do is apply the indexing function to the literal expression 5 and 
%we get a description of how to compute that value. 

%This array representation have been working fairly well in many situations. 
%In {\tt example1} where we used {\tt zipWith} we got just about the resulting
%CUDA code that a CUDA programmer would write himself. For some other examples 
%the situation is much worse. 


The previous subsection described positive aspects of the array 
representation that we have used so far. There are, however, 
circumstances in which this Array representation is too restricted. 

Take the problem of concatenating two arrays. Using the array representation 
described above, the only way to concatenate two arrays is to introduce 
a conditional into the indexing function. If {\tt f} and {\tt g} are the 
indexing functions of two arrays that are to be concatenated, and {\tt n1}
is the length of the first array, the indexing 
function of the result must be
\begin{codesize}
\begin{verbatim}
new ix = if (ix < n1) 
         then f ix 
         else g (ix - n1)
\end{verbatim}
\end{codesize}
The following program
concatenates two arrays: 
\begin{codesize}
\begin{verbatim}
catArrays :: (Array IntE, Array IntE) -> Kernel (Array IntE)
catArrays = pure conc
\end{verbatim}
\end{codesize}
%(arr1,arr2) = return$ conc (arr1,arr2)
When it is used to generate a CUDA kernel that concatenates
two arrays of length $16$, the following code is the result: 
\begin{codesize}
\begin{verbatim}
__global__ void catArrays(int *input0,int *input1,int *result0){
  unsigned int tid = threadIdx.x;
  unsigned int bid = blockIdx.x;
  
  result0[((bid*32)+tid)] = 
    (tid<16) ? input0[((bid*16)+tid)] : 
               input1[((bid*16)+(tid-16))];
}
\end{verbatim}
\end{codesize}
%__global__ void catArrays(int *input0,
%                          int *input1,
%                          int *result0){
%  unsigned int tid = threadIdx.x;
%  unsigned int bid = blockIdx.x;
%  
%  result0[((bid*64)+tid)] = 
%    (tid<32) ? input0[((bid*32)+tid)] : 
%               input1[((bid*32)+(tid-32))];
%  
%}
Now, conditionals like these are {\em bad} in code 
to execute on a GPU, with its wide-SIMD data-parallel model. Separating
the operation into two assignments and using half as many threads gives
much higher performance.
\begin{codesize}
\begin{verbatim}
__global__ void catArraysByHand(int *input0,
                                int *input1,
                                int *result0){
  unsigned int tid = threadIdx.x;
  unsigned int bid = blockIdx.x;
  
  result0[((bid*32)+tid)] = input0[((bid*16)+tid)];
  result0[((bid*32)+tid+16)] = input1[((bid*16)+tid)];  
}
\end{verbatim}
\end{codesize}
There are cases where code with conditionals is not that bad. 
An expert on NVIDIA GPUs in particular may say that code with 
the condition {\tt (tid < 32)} is fine, since 32 is the SIMD width 
of those GPUs. However, any number that is not a multiple of 32 would 
lead to poor performance, so in general this is a problem. 
 Worse still, {\em zipping} two arrays together and then 
{\em unpairing} (to get an array of elements)
leads to code that takes two different paths depending 
on odd or even {\em thread id}. When a GPU executes such code, it 
shuts down half of the threads and computes the two paths in sequence. 
\begin{codesize}
\begin{verbatim}
zippUnpair :: (Array IntE, Array IntE) -> Kernel (Array IntE) 
zippUnpair = pure (unpair . zipp)
\end{verbatim}
\end{codesize}
The {\tt zipp} and {\tt unpair} operations are defined as follows: 
\begin{codesize}
\begin{verbatim}
zipp :: (Array a, Array b) -> Array (a, b)             
zipp (arr1,arr2) = 
     Array (\ix -> (arr1 ! ix, arr2 ! ix)) 
           (min (len arr1) (len arr2))

unpair :: Choice a => Array (a,a) -> Array a
unpair arr = 
    let n = len arr
    in  Array (\ix -> ifThenElse ((mod ix 2) ==* 0) 
                      (fst (arr ! (ix `shiftR` 1)))
                      (snd (arr ! (ix `shiftR` 1)))) (2*n)
\end{verbatim}
\end{codesize}
Code generated from the {\tt zippUnpair} program exhibits really 
poor performance; at any time half of the threads are shut down. 
\begin{codesize}
\begin{verbatim}
__global__ void zippUnpair(int *input0,
                           int *input1,
                           int *result0){
  unsigned int tid = threadIdx.x;
  unsigned int bid = blockIdx.x;
  
  result0[((bid*64)+tid)] = 
    ((tid%2)==0) ? input0[((bid*32)+(tid>>1))] : 
                   input1[((bid*32)+(tid>>1))];
}
\end{verbatim}
\end{codesize}
If we wrote this CUDA program by hand, we would, again, split it up into 
two phases so that all threads can progress in parallel.  

%Even though this is a pretty artificial example, it illustrates the 
%problem well. The situation is just as bad when using the function
%{\tt unpair}. This function takes an array of pairs and returns an 
%array of elements such that position 0 and 1 in the result array holds 
%the values of the pair at index 0 in the input array. Using this operation 
%results in a conditional statement that takes different execution paths 
%on odd and even numbered threads. 
  
The arrays described so far, with an indexing function and a length, have 
been nicknamed {\em Pull arrays} for how they describe how to 
compute an element by {\em pulling} data from a number 
of places. Using just Pull arrays, we have been unable to solve the 
problems described so far in this section. The solution is to add a complementary array type to Obsidian. 

%For a more in depth view of the parts of Obsidian that has been showed 
%in this section see~\cite{JSLIC}.








