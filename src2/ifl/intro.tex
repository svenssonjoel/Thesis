\section{Introduction}

There is a pressing need for new and better ways to program parallel
machines. Graphics Processing Units (GPUs) are one kind of such parallel
machines that provide a lot of computing power. Modern GPUs have become
extremely interesting for {\em general purpose} programming, i.e. computing
functions that have little or nothing to do with graphics programming.

We aim to develop high-level methods and tools, based on functional
programming, for low-level general purpose programming of GPUs. The methods
are high-level because we are making use of common powerful programming
abstractions in functional programming, such as higher-order functions and
polymorphism. At the same time, we still want to provide control to the
programmer on important low-level details such as how much parallelism is
introduced, and memory layout. (These aims are in contrast to other
approaches to GPU programming, where the programmer expresses the intent in
a high-level language, and then lets a smart compiler try to do its best.)

Based on our earlier work on structural hardware design, we plan to
investigate whether or not a {\em structure-oriented programming style} can
be used in programming modern GPUs. We are developing Obsidian, a domain
specific language embedded in Haskell. The aim is to make extensive use of
higher order functions capturing {\em connection patterns}, and from these
compact descriptions to generate code to run on the GPUs. Our
hardware-oriented view of programming also leads us to investigate the use
of algorithmic ideas from the hardware design community in parallel
programming.

In the rest of the paper, we first briefly review our earlier work on
hardware description languages (Sect.\ \ref{sec:combinators}), and introduce
the GPU architecture we are working with (Sect.\ \ref{sec:gpu}). After
that, we describe the current status of the language Obsidian, and show
examples (Sect.\ \ref{sec:obsidian}) and experimental results (Sect.\
\ref{sec:results}). Obsidian is currently very much work in progress; we
discuss motivations and current shortcomings (Sect.\ \ref{sec:disc}) and
future directions (Sect.\ \ref{sec:fut}).
