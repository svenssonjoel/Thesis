% ------------------------------------------------------------------------------
\subsection{Future work} \label{sec:fut}

The version of Obsidian described here is at a very experimental stage. 
The quality of the C code generated needs to improve to get performance 
on par with the previous version. The previous version however, was very
limited in what you could express. This older version is described in
\citeifl{JMT}.
There is a clear opportunity to perform classic compiler optimisations on 
the IC formed by running an Obsidian program. Currently this is not done at all. 

Ways to describe the coordination 
of kernels in code that is still short and sweet are also needed. Some experiments using 
methods similar to Lava's netlist generation have been performed, but the
resulting performance is not yet satisfactory. In CUDA, Kernel coordination 
is in part described in the actual kernel code. Kernels decide which parts 
of the given data to use. As future work we will approach the kernel 
coordination problem at a lower more CUDA-like level. We will, of necessity,
need to develop programming idioms or combinators that express how data is 
placed in the memory hierarchy. The isolation of this as a central question 
is one of the more unexpected and interesting results of the project.
Right now work is focused on developing combinators that are more clever 
in their treatment of {\tt sync}s. This leads to new data structures that 
allow the merging of {\tt sync}s. This new approach seems to make efficient
implementations of combinators such as {\tt two} possible. 



%We also need to experiment with ways of ``moving syncs around'' in our 
%generated code, in order to avoid unnecessary serialisation caused by having 
%too many syncs of small groups of threads.
